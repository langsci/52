\addcontentsline{toc}{chapter}{Postscriptum}
\chapter*{Postscriptum}
\label{c:conclusion}

This book is a trimmed version of my doctoral dissertation in which I investigated how case systems may emerge as the consequence of locally situated interactions in a population of autonomous artificial agents that shape and reshape their language in order to optimize their communicative success \citep{vantrijp08analogy}. When I submitted my thesis, I felt that I had just finished the first part of a longer saga, and the past six years have proven that feeling to be right. This postscriptum therefore summarizes the directions that my research has taken since 2008.

\section*{Fluid Construction Grammar}

If there is one important aspect that sets the experiments in this book apart from previous experiments in artificial language evolution, it is the fact that they are more strongly connected to empirical evidence of real-life language evolution. Earlier experiments typically involved abstract models, whereas I {\em reverse-engineered} a processing model of English argument structure constructions in Fluid Construction Grammar. This methodological innovation\footnote{I do not claim to be the inventor of this innovation: the choice of reverse-engineering an actual language was in the first place made possible by the vision of Luc Steels, who realized that more sophisticated language technologies were required for moving the field of artificial language evolution forward \citep{steels04constructivist}, and by my colleagues (particularly Joachim De Beule, Martin Loetzsch, Michael Spranger and Pieter Wellens) who further developed Steels' implementation into the FCG-system and experimental framework that support the experiments in this book \citep{loetzsch:08b}.} ensures that the agents have sufficiently sophisticated representation and processing techniques for handling linguistic structures of natural language-like complexity, and it offers a `target structure' that helps the experimenter to identify adequate learning and innovation operators. It also led to the first computational and bidirectional construction grammar implementation of argument structure \citep{vantrijp08argumentsstruktur}, which demonstrates that it is perfectly feasible to operationalize constructional analyses in a formally precise way.

The formalization of argument structure in this book stretched the expressive power of the `2005-2007-implementation' of FCG. My colleagues and I therefore came together in a groundbreaking FCG workshop in Ellezelles (Belgium, 30 June -- 4 July 2008) in which almost all of FCG's present-day features were devised and implemented.\footnote{The workshop participants included Luc Steels, Joris Bleys, Thomas Cederborg, Pascal Costanza, Joachim De Beule, Katja Gerasymova, Martin Loetzsch, Vanessa Micelli, Simon Pauw, Michael Spranger, Pieter Wellens and myself.} FCG is now regarded as an innovative and mature grammar formalism \citep{vantrijp13comparison} that has been applied for reverse-engineering grammars of English, French, Russian, German, Hungarian, Polish, Spanish, and so on \citep{steels11design,steels12computational,steels12language}. Many of those grammars have served as a basis for understanding the evolution of intricate phenomena such as colours \citep{bleys10}, spatial terms \citep{spranger:11b} and quantifiers \citep{pauw13}.

The increased expressive power of FCG has allowed me to tackle many non-trivial problems concerning argument structure and language processing. First of all, this book's proposal for handling argument structure has turned out to be a recurrent `design pattern' for argument realization and has been further refined by \citet{vantrijp11design}. The implementation has been extended with solutions for feature indeterminacy and ambiguity \citep{vantrijp11feature}, and long-distance dependencies \citep{vantrijp14}; and it has been grounded on humanoid robots \citep{steels12action}. I have also been particularly concerned with fluid and robust language processing \citep{steels11how}, integrating diagnostics and repairs in the FCG-system \citep{beuls12diagnostics} and exploring reflective architectures for open-ended processing \citep{vantrijp12robust}.

\section*{Artificial Language Evolution}

Most experiments in artificial language evolution involve direct one-to-one mappings between meaning and form. The experiments reported in this book have significantly pushed the state-of-the-art by showing how polysemous categories may emerge in a multi-agent population \citep[for more recent results, see][]{vantrijp10grammaticalization,vantrijp:11a,vantrijp12b,vantrijp12case}. Moreover, the experiments have identified multi-level selection as a crucial step in the transition from lexical to grammatical languages. The relation between multi-level selection and language systematicity has been explored in more detail by \citet{vantrijp:12f}.

The experiments have also taken an exciting turn in recent years by applying the model to real-life language phenomena, which is made possible thanks to the aforementioned advances in Fluid Construction Grammar. The first experiment of this kind is presented by \citet{vantrijp10spanish}, who investigates an ongoing evolution in the pronoun system of Spanish. More specifically, the experiment demonstrates how a population of language users are able to shift a case-based system of pronouns to a gender-based system without loss in communicative success (despite competing variants in the population).

Another recent case study focuses on German definite articles \citep{vantrijp:12e,vantrijp:12c,vantrijp2013ldc,vantrijp14fitness}, which are notorious for their case syncretism (i.e. the same form maps onto multiple, often conflicting functions). These syncretic forms have long been regarded as non-systematic, historical accidents. The agent-based models however demonstrate that the system of definite articles has evolved to become easier to process by comparing a reverse-engineered processing model of the current German grammar to a model of its oldest attested historical predecessor (Old High German).

\addcontentsline{toc}{chapter}{Acknowledgements}
\section*{Acknowledgements}

And finally I arrive at the last sentences of this book, and undoubtedly the most difficult ones to write because here I have to find the right words to express my gratitude to so many people who have supported me through all these years.

The first person I want to thank is Luc Steels, founder of the SONY Computer Science Laboratory Paris, and the best mentor imaginable for a young researcher such as myself. Not only has he provided me with a secure position in a superb research environment, he also still manages to surprise me with his groundbreaking ideas and his relentless energy. Luc is able to write a research proposal in the morning, conduct an experiment in the afternoon, and compose an opera in the evening. I hope that one day I can discover his secret. I also wish to thank Walter Daelemans and Guy De Pauw for their guidance and input; and for introducing me to Luc and thereby landing me my first job in science. I always enjoy my visits to Antwerp for the interesting discussions and fresh perspective.

In the past eight years, I have been extremely fortunate to work with some of the brightest people I have ever met. So I would like to thank my colleagues in Paris (in order of appearance): Benjamin K. Bergen, Martin Loetzsch, Wouter Van den Broeck, Michael Spranger, Vanessa Micelli, Katja Gerasymova, Simon Pauw, Damien Munch, Nancy Chang, Manfred Hild, Fabrizio Lo Scudo, Miquel Cornudella Gaya and Paul Van Eecke; as well as Sophie Boucher, Peter Hanappe, Fr\'{e}d\'{e}ric Kaplan, Pierre-Yves Oudeyer, Fran\c{c}ois Pachet, Pierre Roy and Nicolas Duval. I would also like to thank my colleagues from Barcelona and Brussels Joachim De Beule, Joris Bleys, Bart De Vylder, Pieter Wellens, Wout Monteyne, Frederik Himpe, Carl Jacobs, Thomas Cederborg, Katrien Beuls, Kevin Stadler, Lara Mennes, Bart de Boer, Em\'{i}lia Garcia Casademont and Yana Knight.

I would be nowhere in life if I hadn't grown up in the most wonderful family one can imagine. I want to thank my parents for working so hard for me, for their love and for giving me all the happiness and opportunities that most people can only dream of. Thanks to my brother and sisters, Bianca, Davy and Tiny; their spouses Paul, Ann and Peter; and their children Lorenz, Stef, Laura, my godchild Zo\"e, Isabo and Amber. Thanks to my family-in-law Flor, Gerda, Griet, Kim, Bart, Joren and Joppe for accepting me in your family and giving me a place to stay each time I need one. Thanks to all my friends and my `arch-nemesis' Filip. Missing all of you has been the hardest part of my life.

\vspace{1cm}
\noindent Finally, I would like to thank the women in my life. First, my lovely wife Katleen, for all her love and support; for sticking with me and putting up with my bad moods and stress; and for always being there for me. Seeing your smile each day is more than any man ever deserves, and I couldn't have done this without you. 
Secondly, my beautiful daughter Elise, who has turned my world upside down and made it so much better in every way. Despite her young age, she never ceases to amaze me with her joy and creativity, and just by being herself she makes me the proudest father on the planet. I dedicate this book to the both of you. Thank you.

\begin{flushright}
Paris, 10 April 2014
\end{flushright}






